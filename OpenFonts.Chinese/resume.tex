%%%%%%%%%%%%%%%%%%%%%%%%%%%%%%%%%%%%%%%
% Deedy - One Page Two Column Resume
% LaTeX Template
% Version 1.2 (16/9/2014)
%
% Original author:
% Debarghya Das (http://debarghyadas.com)
%
% Original repository:
% https://github.com/deedydas/Deedy-Resume
%
% IMPORTANT: THIS TEMPLATE NEEDS TO BE COMPILED WITH XeLaTeX
%
% This template uses several fonts not included with Windows/Linux by
% default. If you get compilation errors saying a font is missing, find the line
% on which the font is used and either change it to a font included with your
% operating system or comment the line out to use the default font.
% 
%%%%%%%%%%%%%%%%%%%%%%%%%%%%%%%%%%%%%%
% 
% TODO:
% 1. Integrate biber/bibtex for article citation under publications.
% 2. Figure out a smoother way for the document to flow onto the next page.
% 3. Add styling information for a "Projects/Hacks" section.
% 4. Add location/address information
% 5. Merge OpenFont and MacFonts as a single sty with options.
% 
%%%%%%%%%%%%%%%%%%%%%%%%%%%%%%%%%%%%%%
%
% CHANGELOG:
% v1.1:
% 1. Fixed several compilation bugs with \renewcommand
% 2. Got Open-source fonts (Windows/Linux support)
% 3. Added Last Updated
% 4. Move Title styling into .sty
% 5. Commented .sty file.
%
%%%%%%%%%%%%%%%%%%%%%%%%%%%%%%%%%%%%%%%
%
% Known Issues:
% 1. Overflows onto second page if any column's contents are more than the
% vertical limit
% 2. Hacky space on the first bullet point on the second column.
%
%%%%%%%%%%%%%%%%%%%%%%%%%%%%%%%%%%%%%%


\documentclass[]{deedy-resume-openfont}
\usepackage{fancyhdr,graphicx}
    
\pagestyle{fancy}
\fancyhf{}
    
\begin{document}

%%%%%%%%%%%%%%%%%%%%%%%%%%%%%%%%%%%%%%
%
%     LAST UPDATED DATE
%
%%%%%%%%%%%%%%%%%%%%%%%%%%%%%%%%%%%%%%
\lastupdated

%%%%%%%%%%%%%%%%%%%%%%%%%%%%%%%%%%%%%%
%
%     TITLE NAME
%
%%%%%%%%%%%%%%%%%%%%%%%%%%%%%%%%%%%%%%
\namesection{廖}{紫默}{ \urlstyle{same}\href{mailto:zimoliao@mail.ustc.edu.cn}{zimoliao@mail.ustc.edu.cn} | 18000512202
}

%%%%%%%%%%%%%%%%%%%%%%%%%%%%%%%%%%%%%%
%
%     COLUMN ONE
%
%%%%%%%%%%%%%%%%%%%%%%%%%%%%%%%%%%%%%%

\begin{minipage}[t]{0.25\textwidth}

    \vspace*{1ex}
    {\includegraphics[width=.75\linewidth]{profile.jpg}}
    \sectionsep

    %%%%%%%%%%%%%%%%%%%%%%%%%%%%%%%%%%%%%%
    %     EDUCATION
    %%%%%%%%%%%%%%%%%%%%%%%%%%%%%%%%%%%%%%

    \section{教育经历}
    \sectionsep

    \subsection{\bf 中国科学技术大学}
    \descript{\bf 硕博连读,流体力学}
    \descript{\bf 导师 陆夕云院士}
    \location{2021.09至今}
    \sectionsep

    \subsection{南京航空航天大学}
    \descript{学士学位,飞行器设计与工程}
    \descript{工科研究试验班}
    \location{2017.09-2021.06}
    \sectionsep

    \subsection{四川省成都市第七中学}
    \location{2014.09-2017.06}
    \sectionsep

    %%%%%%%%%%%%%%%%%%%%%%%%%%%%%%%%%%%%%%
    %     COURSEWORK
    %%%%%%%%%%%%%%%%%%%%%%%%%%%%%%%%%%%%%%

    \section{修读课程}
    \sectionsep    % \subsection{研究生}
    高等应用数学(94) \\
    计算流体力学(98) \\
    流动稳定性和湍流(99) \\
    GPU并行计算(99) \\
    \sectionsep

    %%%%%%%%%%%%%%%%%%%%%%%%%%%%%%%%%%%%%%
    %     SKILLS
    %%%%%%%%%%%%%%%%%%%%%%%%%%%%%%%%%%%%%%

    \section{技能}
    \sectionsep
    \subsection{专业}
    % \location{超过 5000 行}
    湍流\textbullet{}多相流\textbullet{}计算流体力学 \\
    数据驱动流体力学\textbullet{}机器学习\\
    \sectionsep
    \subsection{编程}
    % \location{超过 5000 行}
    C/C++\textbullet{}Fortran\textbullet{}MPI \\
    CUDA C\textbullet{}CUDA Fortran \\
    MATLAB\textbullet{}Python\textbullet{}\LaTeX \\
    \sectionsep

    %%%%%%%%%%%%%%%%%%%%%%%%%%%%%%%%%%%%%%
    %     LINKS
    %%%%%%%%%%%%%%%%%%%%%%%%%%%%%%%%%%%%%%

    \section{链接}
    \sectionsep
    Zhihu://  \href{https://www.zhihu.com/people/lzmo}{\bf ANFLK} \\
    (4.6k+ 收藏,3.6k+ 关注者) \\
    ResearchGate:// \href{https://www.researchgate.net/profile/Zimo_Liao}{\bf ZimoLiao} \\
    \sectionsep

    %%%%%%%%%%%%%%%%%%%%%%%%%%%%%%%%%%%%%%
    %
    %     COLUMN TWO
    %
    %%%%%%%%%%%%%%%%%%%%%%%%%%%%%%%%%%%%%%

\end{minipage}
\hfill
\begin{minipage}[t]{0.73\textwidth}

    %%%%%%%%%%%%%%%%%%%%%%%%%%%%%%%%%%%%%%
    %     RESEARCH
    %%%%%%%%%%%%%%%%%%%%%%%%%%%%%%%%%%%%%%

    \section{一作论文与专利}
    \sectionsep
    \runsubsection{\href{https://doi.org/10.1017/jfm.2023.435}{\bf Reduced-order variational mode decomposition to reveal transient and non-stationary dynamics in fluid flows}}

    \descript{Journal of Fluid Mechanics,流体力学顶刊}
    \location{2023.06}
    \vspace*{1ex}
    \begin{itemize}\itemsep1pt \parskip0pt \parsep0pt
        \item 基于信号处理领域中的变分模态分解,提出了降阶变分模态分解(RVMD)方法
        \item RVMD可以{\bf 自适应地从高维、瞬态/非平稳的时空数据中提取低维动力学过程}
        \item 该方法{\bf 有望进一步应用于其他复杂系统、神经动力学、金融时间序列分析等领域}
        \item 其MATLAB实现已在Github开源:\href{https://github.com/ZimoLiao/rvmd}{https://github.com/ZimoLiao/rvmd}
    \end{itemize}
    \sectionsep

    \runsubsection{\href{https://doi.org/10.1016/j.ijmultiphaseflow.2024.104840}{\bf GPU acceleration of four-way coupled PP-DNS for compressible particle-laden wall turbulence}}

    \descript{International Journal of Multiphase Flow,多相流顶刊}
    \location{2024.06}
    \vspace*{-1ex}
    \begin{itemize}\itemsep1pt \parskip0pt \parsep0pt
        \item 提出了包括{\bf 数据结构、MPI通讯策略的气粒两相流计算完整算法流程}
        \item 编程实现了大规模、四向耦合气粒两相湍流的GPU加速数值模拟
        \item 可以{\bf 在8张NVIDIA A100卡上完成10的9次方量级颗粒/网格数的高效计算}
    \end{itemize}
    \sectionsep


    \runsubsection{{\bf 一种可压缩气粒两相壁湍流的GPU/CPU异构并行计算方法}}

    \descript{发明专利,202311211287.8}
    \location{2023.12}
    \sectionsep

    
    \runsubsection{{\bf The transport map of particle-laden channel flow}}

    \descript{Journal of Fluid Mechanics,在投}
    \location{2024.07}
    \vspace*{-1ex}
    \begin{itemize}\itemsep1pt \parskip0pt \parsep0pt
        \item 基于湍流统计理论和气体动理论给出了气粒两相流中质量/动量/能量的完整输运图谱
        \item 结合高保真数值模拟结果揭示了惯性颗粒对壁湍流的调制规律与机理
    \end{itemize}

    % \runsubsection{{ Temperature and heat flux bounds of convection driven by non-uniform internal heating}}

    % \descript{Acta Mechanica Sinica,三作}
    % \sectionsep

    % \runsubsection{{ Effects of inflow Mach numbers on shock train dynamics and turbulence features in a backpressured supersonic channel flow}}

    % \descript{Physics of Fluids,三作}
    % \sectionsep

    %%%%%%%%%%%%%%%%%%%%%%%%%%%%%%%%%%%%%%
    %     OPEN SOURCE
    %%%%%%%%%%%%%%%%%%%%%%%%%%%%%%%%%%%%%%

    \section{项目经历}
    \begin{tabular}{ll}
        {\color{date}\bf 参与项目名称/类型}                     & {\color{date}\bf 个人贡献}  \vspace*{1ex}\\
        {\bf 极端流动的多过程问题研究}                     & 高速气粒两相壁湍流模拟分析  \\
        {\color{subheadings} 国家自然科学基金基础科学中心项目} & \vspace*{1ex}  \\
        {\bf 高超声速飞行器转捩/湍流结构与降热减阻集成研究}          & 激波串非定常特性分析     \\
        {\color{subheadings} 国家自然科学基金重大研究计划集成项目} & \vspace*{1ex}  \\
        {\bf 固体火箭发动机***数值模拟研究}                 & 喷管气粒两相流模拟分析与建模 \\
        {\color{subheadings} ***基础研究项目群}       & \vspace*{1ex}  \\
    \end{tabular}
    \sectionsep

    %%%%%%%%%%%%%%%%%%%%%%%%%%%%%%%%%%%%%%
    %     AWARDS
    %%%%%%%%%%%%%%%%%%%%%%%%%%%%%%%%%%%%%%

    \section{奖项与荣誉}
    \begin{tabular}{ll}
        \bf 特等奖(第一名) & \bf 第十二届全国周培源大学生力学竞赛(个人赛)        \\
        \bf 特等奖      & \bf 第十二届全国周培源大学生力学竞赛“理论设计与操作”团体赛 \\
        奖学金      & 本科生国家奖学金                     \\
    \end{tabular}
    \sectionsep

    %%%%%%%%%%%%%%%%%%%%%%%%%%%%%%%%%%%%%%
    %     EXPERIENCE
    %%%%%%%%%%%%%%%%%%%%%%%%%%%%%%%%%%%%%%

    \section{其他}
    % \sectionsep
    \begin{tightemize}
        \item {\bf 基于有限差分方法、格子玻尔兹曼方法,采用C/C++,Fortran,CUDA等语言独立\\编写过多套CPU/GPU流体力学求解器}
        \item 中学至本科竞赛经历丰富,{\bf 具有很强的学习、解决问题的能力,擅长交叉学科研究}
        \item 本科期间担任校普通生排球队主力二传,身体素质不错
    \end{tightemize}
    \sectionsep

    % \runsubsection{摩根士丹利}
    % \descript{CIP 项目实习生}
    % \location{2017.02-2017.08 | 上海}
    % \begin{tightemize}
    % \item 优化开源容器调度管理框架 treadmill 的调度器
    % \item 实现与 Kubernetes 类似的调度模型,同时保留自身的树形结构
    % \end{tightemize}
    % \sectionsep

    % \runsubsection{上海触宝信息技术有限公司}
    % \descript{数据工程师(实习)}
    % \location{2015.09-2015.09 | 上海}
    % \begin{tightemize}
    % \item 移植爬虫代码到新的平台,优化重写部分过期的爬虫
    % \end{tightemize}
    % \sectionsep

    % \runsubsection{蚂蚁金服(杭州)网络技术有限公司}
    % \descript{Java 研发工程师(实习)}
    % \location{2015.07-2015.09 | 杭州}
    % \begin{tightemize}
    % \item 在支付宝国际事业团队从事海外直购业务开发
    % \item 实现部分包裹清关的逻辑和后台管理的逻辑
    % \end{tightemize}
    % \sectionsep

    %%%%%%%%%%%%%%%%%%%%%%%%%%%%%%%%%%%%%%
    %     PUBLICATIONS
    %%%%%%%%%%%%%%%%%%%%%%%%%%%%%%%%%%%%%%

    % \section{Publications} 
    % \renewcommand\refname{\vskip -1.5cm} % Couldn't get this working from the .cls file
    % \bibliographystyle{abbrv}
    % \bibliography{publications}
    % \nocite{*}

\end{minipage}
\end{document}  \documentclass[]{article}
